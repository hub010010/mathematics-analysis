% Options for packages loaded elsewhere
\PassOptionsToPackage{unicode}{hyperref}
\PassOptionsToPackage{hyphens}{url}
%
\documentclass[
]{article}
\usepackage{amsmath,amssymb}
\usepackage{iftex}
\ifPDFTeX
  \usepackage[T1]{fontenc}
  \usepackage[utf8]{inputenc}
  \usepackage{textcomp} % provide euro and other symbols
\else % if luatex or xetex
  \usepackage{unicode-math} % this also loads fontspec
  \defaultfontfeatures{Scale=MatchLowercase}
  \defaultfontfeatures[\rmfamily]{Ligatures=TeX,Scale=1}
\fi
\usepackage{lmodern}
\ifPDFTeX\else
  % xetex/luatex font selection
\fi
% Use upquote if available, for straight quotes in verbatim environments
\IfFileExists{upquote.sty}{\usepackage{upquote}}{}
\IfFileExists{microtype.sty}{% use microtype if available
  \usepackage[]{microtype}
  \UseMicrotypeSet[protrusion]{basicmath} % disable protrusion for tt fonts
}{}
\makeatletter
\@ifundefined{KOMAClassName}{% if non-KOMA class
  \IfFileExists{parskip.sty}{%
    \usepackage{parskip}
  }{% else
    \setlength{\parindent}{0pt}
    \setlength{\parskip}{6pt plus 2pt minus 1pt}}
}{% if KOMA class
  \KOMAoptions{parskip=half}}
\makeatother
\usepackage{xcolor}
\setlength{\emergencystretch}{3em} % prevent overfull lines
\providecommand{\tightlist}{%
  \setlength{\itemsep}{0pt}\setlength{\parskip}{0pt}}
\setcounter{secnumdepth}{-\maxdimen} % remove section numbering
\ifLuaTeX
  \usepackage{selnolig}  % disable illegal ligatures
\fi
\usepackage{bookmark}
\IfFileExists{xurl.sty}{\usepackage{xurl}}{} % add URL line breaks if available
\urlstyle{same}
\hypersetup{
  hidelinks,
  pdfcreator={LaTeX via pandoc}}

\author{}
\date{}

\begin{document}

\section{第14章
数项级数}\label{ux7b2c14ux7ae0--ux6570ux9879ux7ea7ux6570}

\subsection{14.1
无穷级数的基本性质}\label{141-ux65e0ux7a77ux7ea7ux6570ux7684ux57faux672cux6027ux8d28}

\tableofcontents

\subsubsection{14.1.1
无穷级数的定义}\label{1411-ux65e0ux7a77ux7ea7ux6570ux7684ux5b9aux4e49}

\[无穷级数:\sum_{n=1}^{\infty}a_n = a_1 +a_2 +\cdots+a_n+\cdots\\
部分和: S_n=\sum_{k=1}^{n}a_k\\
级数收敛: \lim_{n\to\infty}S_n=S存在\\ 
否则,级数发散\]

\subsubsection{14.1.2
收敛的必要条件}\label{1412-ux6536ux655bux7684ux5fc5ux8981ux6761ux4ef6}

\[\lim_{n\to\infty}a_n=0\]

\subsubsection{14.1.3
收敛级数的可加/可数乘性}\label{1413-ux6536ux655bux7ea7ux6570ux7684ux53efux52a0ux53efux6570ux4e58ux6027}

\begin{itemize}
\item
  !!! 前提 !!! : 两级数都收敛

  \[\sum_{k=1}^{\infty}a_k,\sum_{k=1}^{\infty}b_k 收敛\\
    则 \sum_{k=1}^{\infty}(\alpha a_k+\beta b_k)=\alpha\sum_{k=1}^{\infty}a_k+\beta\sum_{k=1}^{\infty}b_k\]
\end{itemize}

\subsubsection{14.1.4
收敛级数的可结合性}\label{1414-ux6536ux655bux7ea7ux6570ux7684ux53efux7ed3ux5408ux6027}

\begin{itemize}
\item
  把 \textbf{收敛级数} 任意 \textbf{结合}
\item
  而 \textbf{不改变} 其 \textbf{先后次序}
\item
  与 \textbf{新级数} 和相同

  \[\sum_{n=1}^{\infty}a_n = (a_1+a_2+\cdots +a_{k_1})+(a_{k_1+1}+\cdots+a_{k_2})+\cdots+(a_{k_{n-1}+1}+\cdots+a_{k_n})+\cdots\\
  新级数: b_n=a_{k_{n-1}+1}+\cdots+a_{k_n}\\
  \sum_{n=1}^{\infty}a_n =\sum_{n=1}^{\infty}b_n\]
\item
  正向:已知 \(\sum_{n=1}^{\infty}a_n\) 收敛,得
  \(\sum_{n=1}^{\infty}b_n\) 收敛
\item
  逆向:已知 \(\sum_{n=1}^{\infty}b_n\) 收敛,且 括号中的项都同号,得
  \(\sum_{n=1}^{\infty}a_n\) 收敛
\end{itemize}

\subsubsection{14.1.5
增加或去掉有限项}\label{1415-ux589eux52a0ux6216ux53bbux6389ux6709ux9650ux9879}

\begin{quote}
\(\sum_{n=1}^{\infty}a_n\) 前 增加或去掉有限项,不影响级数的敛散性
\end{quote}

\end{document}
